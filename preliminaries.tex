\documentclass{book}
\usepackage{amsthm}
\usepackage{amssymb}
\usepackage{appendix}
\usepackage{titlesec}
\usepackage{amsmath}
\usepackage{graphicx}
\usepackage[shortlabels]{enumitem}
\usepackage[dvipsnames]{xcolor}
\usepackage{tikz}
\usetikzlibrary{cd, fit, shapes.geometric, positioning}
\usepackage{pgfplots}
\pgfplotsset{compat=1.18}

\usepackage{hyperref}

\theoremstyle{definition}
\newtheorem{question}{Question}
\newtheorem*{definition*}{Definition}
\newtheorem{exercise}{Exercise}[chapter]
\newtheorem*{exercise*}{Answer}
\newtheorem{answer}{Answer}[chapter]

\newtheoremstyle{colonstylebf}{}{}{\normalfont}{}{\bfseries}{:}{.5em}{}
\theoremstyle{colonstylebf}
\newtheorem*{question*}{Q}

\newcommand{\set}[1]{\{#1\}}
\newcommand{\N}{\mathbb{N}}
\newcommand{\spvdots}{\hspace{5pt} \vdots \hspace{5pt} }
\newcommand*{\levelheight}{11mm}
\newcommand*{\bigdots}{{\Large $\vdots$}}

\titleformat{\subsubsection}[runin]
  {\normalfont\normalsize\bfseries} % Format of the title
  {\thesubsubsection} % Label
  {.5em} % Separation between label and title text
  {} % Before-code
  [] % After-code
  
  \titleformat{\title}[block]
  {\normalfont\Huge\scshape}
  {\thetitle}
  {1em}
  {}
  []

\begin{document}
\chapter{Preliminaries}\label{prelim}
Hey, a heads up: this section includes exercises, and it is expected that you do them. But don't worry;\footnote{Be happy.} the exercises are short and simple and their solutions are in Appendix \ref{answers}. Give each exercise an honest attempt before consulting its solution, though; you'll learn more this way, and you'll learn faster.

\section{A half-start}
\begin{quote}
\emph{Natural numbers were created by God, everything else is the work of men} — Kronecker (1823–1891)
\end{quote}
You might not think about it like this, but the very first thing you learned in math was how to count---as in ``there are three
apples over here and two over there, and if we put them together, there are five apples.'' The numbers involved in this basic form of counting are aptly called the \emph{natural numbers}. One is a natural number, and so is two and three and four and so on.\footnote{9001 is a natural number, too.}

But, as almost everyone already knows, these aren't the only numbers. Some numbers are \dots \emph{\ \!unnatural}, if history is any measure. Zero and negative numbers, for instance, were adopted well after the naturals. And it isn't too hard to imagine why: neither of them can be thought of in the same straight-forward, tangible way as a natural number (good luck pointing at zero or negative three apples). It is necessary to think about these numbers a little differently; we need to recognize that each is a unique leap in abstraction: zero represents the absence of a quantity rather than its multiplicity; and a negative number (in its most straight-forward light) represents a change (a \emph{reduction}) in number rather than a number outright.

The history of now-familiar numbers is fascinating but not the focus of what follows, the bare minimum ideas for pondering the puzzles. The hope of this mini historical detour was to shed light on how even the simplest ideas are worth revisiting, that in doing so we might see them in a new light and gain a new appreciation for them. What follows is a good deal of ``basic'' math which I hope the experienced and uninitiated alike will consider and respect rather than skim or skip. As G.\ P\'olya puts it, ``the advanced reader who skips parts that appear too elementary may miss more than the less advanced reader who skips parts that appear too complex.''

\section{Sets}
A set is a collection of distinct items or ``elements.'' Sets are written by listing their elements inside curly braces and can be represented by variables. Here are some example sets:
\begin{enumerate}[label = (\arabic*)]
\item $\set{terrier, poodle, Godzilla}$,
\item\label{set2} $A = \set{a, \dots, z}$,
\item\label{set3} $\N = \set{1, 2, 3, \dots}$.
\end{enumerate}
Ellipses `{$\dots$}' indicate implied elements that are tedious or impossible to write explicitly, as in items \ref{set2} and \ref{set3}, respectively. These examples also show that sets can be finite or infinite.\footnote{In fact, there are---in an abstract but rigorous sense---infinitely many and ever-growing kinds of infinities. But this is beyond our scope. For more on this, look up Cantor's Theorem.} The number of elements in a set $S$, known as its cardinality, is denoted $|S|$. For example, $|A| = 26$.

\begin{exercise}\label{distinct}
What is $|\{3, 3, 1, 2, 1, 3\}|$?
\end{exercise}

\subsection{The Cartesian product}
The Cartesian product of two sets $S$ and $T$ is the set of all ordered pairs with the first element from $S$ and the second from $T$, denoted $S \times T$.
For example, \[\set{1, 2} \times \set{a, b, c} = \set{(1, a), (1, b), (1, c), (2, a), (2, b), (2, c)}.\]
\begin{exercise}\label{cartesianques}
For finite $S$ and $T$, what is $|S \times T|$ in terms of $|S|$ and $|T|$?
\end{exercise}

\subsection{Bounds of sets}
A set may have lower (or upper) bounds when its elements can be ordered. For example, consider the set $H = \set{0, 1, 2, 3}$. The number $-100$ is a lower bound of $H$ because it is less than or equal to every element of $H$. However, there are greater lower bounds, such as $-99$, $-98$, and so on.
\begin{exercise}\label{lowerbound}
What is the greatest lower bound of $H$? Why?
\end{exercise}
We call the greatest lower bound of a set its \textit{infimum}. The minimum of a set is very similar: it is the least element \emph{contained} in a set.
\begin{exercise}\label{reallower}
Are the infimum and minimum of a set always the same? If they aren't, give an example.
\end{exercise}

\section{Binary Trees}
Figure \ref{tree}: the circles are called nodes, and the lines connecting them are called edges. The topmost node is called the root, and the bottom nodes are called leaves. The root has two descendants (hence the adjective ``binary''), each of which has two descendants of their own and so on. The depth of a tree is the number of edges between the root and any of its leaves; this tree's depth is four.

\begin{figure}[ht]
\centering
\begin{tikzpicture}
[
    scale = .65,
    level 1/.style={sibling distance=60mm, level distance = \levelheight},
    level 2/.style={sibling distance=30mm, level distance = \levelheight},
    level 3/.style={sibling distance=15mm, level distance = \levelheight},
    level 4/.style={sibling distance=7.5mm, level distance = \levelheight},
    state/.style={circle,minimum size =.35cm, draw}
]
	\node[state] {}
		child {node[state] {}
			child {node[state] {}
				child {node[state]{}
					child {node[state]{}}
					child {node[state]{}}
				}
				child {node[state]{}
					child{node[state]{}}
					child{node[state]{}}
				}
			}
			child {node[state] {}
				child {node[state]{}
					child {node[state]{}}
					child {node[state]{}}
				}
				child {node[state]{}
					child{node[state]{}}
					child{node[state]{}}
				}
			}
		}
		child {node[state] {}
		    	child {node[state] {}
				child {node[state]{}
					child {node[state]{}}
					child {node[state]{}}
				}
				child {node[state]{}
					child{node[state]{}}
					child{node[state]{}}
				}
			}
		    	child {node[state] {}
				child {node[state]{}
					child {node[state]{}}
					child {node[state]{}}
				}
				child {node[state]{}
					child{node[state]{}}
					child{node[state]{}}
				}
			}
		};
\end{tikzpicture}

\caption{An example binary tree.}
\label{tree}
\end{figure}
\begin{exercise}
How many leaves does a binary tree of depth $n$ have?
\end{exercise}

\section{Probability}
An event is \emph{random} when it cannot be predicted with complete certainty. The probability of an event $E$ is its likelihood of occurring, denoted $P(E)$. For example, the probability of rolling a 6 on a standard six-sided die (denoted ``d6'') is 1/6.

If an event is guaranteed to happen, its probability is 1. So for any event $E$, $0 \leq P(E) \leq 1$.
\begin{exercise}
Why? \emph{Hint:} This is not a trick question.
\end{exercise}
For convenience, we can denote the outcome of random events with variables. For example, let $D_n$ be the roll of a d$n$ (a standard $n$-sided die); then, again, $P(D_6 = 6) = 1/6$.

However, finding the probability of an event can be challenging when the context is unclear (e.g., what is the probability you think of the number $42$?). Therefore, it's helpful to keep in mind the set of all possible outcomes (relevant to an event), called the \emph{sample space} and denoted $\Omega$. For example, $\Omega = \set{1, 2, 3, 4, 5, 6}$ when rolling a d6: each number represents the outcome that the d6 rolls that value.

When $\Omega$ is finite and each event is equally likely, finding probabilities usually reduces to counting outcomes. Specifically:
\begin{equation}\label{countprob}
P(E) = \frac{\text{number of outcomes where $E$ occurs}}{\text{total number of outcomes}}.
\end{equation}
\begin{exercise}
What is $P(D_6 \text{ is prime})$?\footnote{A prime number is only divisible by 1 and itself. But the first prime number is 2, not 1, because excluding 1 simplifies many theorems about primes.}
\end{exercise}

If outcomes are not equally likely, we can modify the sample space by assigning each event a number of outcomes proportional to its likelihood. For instance, if we have a weighted d4 where each number $x$ comes up $(10\cdot x)\%$ of the time, this would be equivalent to a d10 with the number 1 on one face, 2 on two faces, and so on. 
\begin{exercise}\label{magiccoin}
  Imagine two coins, each with sides labelled 1 and 2. One coin, $F$, is fair,\footnote{(meaning each outcome is equally likely)} while the other one, $M$, is half-magic: if $F = 1$, then $M = 1$; otherwise, $M$ is fair. You flip $F$, then $M$. Describe a weighted die whose outcomes match the random variable $F + M$.

\noindent\emph{Hint}: apply equation \eqref{countprob} and the preceding paragraph.
\end{exercise}


\subsubsection*{Independence.}
Two random events are \emph{independent} if neither gives information about the likelihood of the other. For example, the first roll of a d6 gives no information about its second roll, so the two rolls are independent, unlike $M$ and $F$ in Exercise \ref{magiccoin}. Independent events are almost always easier to reason about than non-independent (i.e., dependent) events.\footnote{Compare the probability that the 100th roll of a d6 is 6 with the probability that the 100th roll of a \emph{special} d6 (identical to a standard one, except if you ever roll more than 20 of a number, it is replaced by the number 1) is 6.}

\subsubsection*{Multiple events.}
Events are usually composed of subevents. There are three fundamental operations that create new events from existing ones.
\begin{enumerate}[label = (\arabic*)]
\item Negation: The event where $E$ does not occur is written $\neg E$.
\item\label{intersection} Intersection:
The event where both $E$ and $F$ occur is written $\displaystyle E \cap F$.
\item Union:  The event where $E$ or $F$ (or both) occur is written $E \cup F$.
\end{enumerate}

\begin{exercise}\label{q:neg}
If $P(E) = p$, what is $P(\neg E)$?
\end{exercise}
\begin{exercise}\label{q:cap}
Suppose events $E$ and $F$ are independent. Can you guess what $\displaystyle P(E \cap F)$ is? \emph{Hint}: Recall the Cartesian product.
\end{exercise}
\begin{exercise}\label{q:cup}
Let $E$ and $F$ be any two events. What is $P(E \cup F)$ in terms of $P(E)$, $P(F)$, and $\displaystyle P(E \cap F)$? \emph{Hint}: Consider the event of rolling an even or prime number with a d6, and draw a Venn diagram.
\end{exercise}

\section{Polynomials}
\subsubsection*{In general.}
A polynomial is an expression made up of indeterminants where the only operations involved are addition, subtraction, and multiplication. For example $x^2 - x - 1$ is a polynomial made up of three terms and one indeterminant. This is a much clearer way of representing ``the product of a value with itself, subtracted by itself and 1,'' wouldn't you say?\footnote{We are lucky: mathematics was largely written in words before Ren\'e Descartes popularized symbolic notation in his ``La G\'eom\'etrie," published in 1637.}

The degree of a polynomial is the highest power of any indeterminant. For example, $x^3 - 7x + y$ has degree 3. A polynomial is called quadratic if its degree is 2.

\subsubsection*{Zeros and quadratic polynomials.}
The zeros of a polynomial are, unsurprisingly, the values of the variable(s) that make the polynomial equal to zero. For example, the zeros of $x^2 - 5x + 6 = (x-3)(x-2)$ are $x = 3$ and $x = 2$. Finding the zeros of a linear (degree 1) polynomial is trivial, but less so for a quadratic.

\begin{exercise}\label{completesquare}
Find a formula for the zeros of a quadratic $p(x) = ax^2 + bx + c$ in terms of its coefficients, $a$, $b$, and $c$.\footnote{The coefficients $a$, $b$, and $c$ are not to be confused with indeterminants: they represent only one possible value, unlike indeterminants that represent several possible values. In the earlier polynomial, $a = 1$, $b = -5$ and $c = 6$.} \emph{Hint}: Can you turn $p(x)$ into a form that looks like $(x + d)^2 = e$?\footnote{$d$ and $e$ are not indeterminants; they are particular values, known as \emph{constants}  (like $a$, $b$, and $c$).}
\end{exercise}
\end{document}
