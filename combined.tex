\documentclass{book}
\usepackage{amsthm}
\usepackage{amssymb}
\usepackage{appendix}
\usepackage{titlesec}
\usepackage{amsmath}
\usepackage{graphicx}
\usepackage[shortlabels]{enumitem}
\usepackage[dvipsnames]{xcolor}
\usepackage{tikz}
\usetikzlibrary{cd, fit, shapes.geometric, positioning}
\usepackage{pgfplots}
\pgfplotsset{compat=1.18}

\usepackage{hyperref}

\theoremstyle{definition}
\newtheorem{question}{Question}
\newtheorem*{definition*}{Definition}
\newtheorem{exercise}{Exercise}[chapter]
\newtheorem*{exercise*}{Answer}
\newtheorem{answer}{Answer}[chapter]

\newtheoremstyle{colonstylebf}{}{}{\normalfont}{}{\bfseries}{:}{.5em}{}
\theoremstyle{colonstylebf}
\newtheorem*{question*}{Q}

\newcommand{\set}[1]{\{#1\}}
\newcommand{\N}{\mathbb{N}}
\newcommand{\spvdots}{\hspace{5pt} \vdots \hspace{5pt} }
\newcommand*{\levelheight}{11mm}
\newcommand*{\bigdots}{{\Large $\vdots$}}

\titleformat{\subsubsection}[runin]
  {\normalfont\normalsize\bfseries} % Format of the title
  {\thesubsubsection} % Label
  {.5em} % Separation between label and title text
  {} % Before-code
  [] % After-code
  
  \titleformat{\title}[block]
  {\normalfont\Huge\scshape}
  {\thetitle}
  {1em}
  {}
  []

\begin{document}
% START OF FILE 'title.tex'
\title{{\centering\Huge\textsc{Assorted Puzzle Solutions}}}
\author{\Large Charles Kolozsvary}
\date{}

\maketitle

\begin{quote}
{\raggedleft {\sl To Nick\/}: \\ 
\hfill {\sl For bringing to my attention the threat of certain crustaceans}}
\end{quote}
% END OF FILE 'title.tex'
% START OF FILE 'preface.tex'
\chapter*{Preface}
I heard from a friend in December 2022 that a company called Jane Street posts a ``puzzle'' each month. But they aren't typical puzzles (like jig-saws or crosswords), they are usually, rather, math-heavy brain teasers. For instance, the December 2022 puzzle, \emph{Die Agony}, goes as follows:
\[
\begin{tikzpicture}
  \filldraw[color=yellow!50] (0,0)--(1,0)--(1,1)--(0,1)--cycle;
  \filldraw[color=blue!50] (5,5)--++(1,0)--++(0,1)--++(-1,0)--cycle;
  \def\drawNumber#1#2#3{\draw node at (#1+.5, #2+.5) {\large \textsf{#3}};}
  \drawNumber{0}{0}{0}
  \drawNumber{1}{0}{77}
  \drawNumber{2}{0}{32}
  \drawNumber{3}{0}{403}
  \drawNumber{4}{0}{337}
  \drawNumber{5}{0}{452}
  
  \drawNumber{0}{1}{5}
  \drawNumber{1}{1}{23}
  \drawNumber{2}{1}{-4}
  \drawNumber{3}{1}{592}
  \drawNumber{4}{1}{445}
  \drawNumber{5}{1}{620}
    
  \drawNumber{0}{2}{-7}
  \drawNumber{1}{2}{2}
  \drawNumber{2}{2}{357}
  \drawNumber{3}{2}{452}
  \drawNumber{4}{2}{317}
  \drawNumber{5}{2}{395}
  
  \drawNumber{0}{3}{186}
  \drawNumber{1}{3}{42}
  \drawNumber{2}{3}{195}
  \drawNumber{3}{3}{704}
  \drawNumber{4}{3}{452}
  \drawNumber{5}{3}{228}
  
  \drawNumber{0}{4}{81}
  \drawNumber{1}{4}{123}
  \drawNumber{2}{4}{240}
  \drawNumber{3}{4}{443}
  \drawNumber{4}{4}{353}
  \drawNumber{5}{4}{508}
  
  \drawNumber{0}{5}{57}
  \drawNumber{1}{5}{33}
  \drawNumber{2}{5}{132}
  \drawNumber{3}{5}{268}
  \drawNumber{4}{5}{492}
  \drawNumber{5}{5}{732}
  
    \draw[line width=2pt] (0,0)--(6,0)--(6,6)--(0,6)--cycle;
    \foreach \x in {1, ..., 5}{
      \draw (\x,0)--(\x,6);
    }
    \foreach \y in {1, ..., 5}{
      \draw (0,\y)--(6,\y);
    }
    \end{tikzpicture}

\]
\begin{quote}
A six-sided die, with numbers written on each of its faces, is placed on the 6-by-6 grid above, in the lower-left (yellow) corner. It then makes a sequence of ``moves''. Each move consists of tipping the die into an orthogonally adjacent square within the grid.

The die starts with a ``score'' of 0. On the $N$th move, its score increases by $N$ times the value of the die facing up after the move. However, the die is only allowed to move into a square if its score after the move matches the value in the square. Also, the die cannot be translated or rotated in place in addition to these moves.

After some number of moves the die arrives in the upper-right (blue) corner.

The answer to this puzzle is the sum of values in the unvisited squares from the die's journey.
\end{quote}

If you think this is cool, great!\footnote{Needless to say, I had a lot of fun solving it; it was especially satisfying to get to apply a bit of group theory I learned during that semester. I realized I could represent the effect of tipping the die in different directions on the arrangement of the die faces with members from $S_6$, the group of permutations with six elements. For example, if the top, east, north, west, south, and bottom faces of the die (from a bird's-eye view perspective) are respectively labelled 1, 2, 3, 4, 5, and 6, tipping the die to the right corresponds to the permutation $(1264)$ in \emph{cycle notation}. I.e., 1 goes to 2, 2 goes to 6, 6 goes to 4, 4 goes to 1, and 3 and 5 don't change. See \url{https://en.wikipedia.org/wiki/Permutation\#Cycle_notation} for more.} The purpose of what follows is to explore many more brain teasers like this and explain their solutions in depth. If you're already fairly comfortable with coding and doing some undergraduate math, I absolutely recommend you
start working on the current Jane Street puzzle (or, if it doesn't suit your fancy, any of the other ones in the archive).

If you're not so familiar with math and computer science but want to hear more interesting problems and their solutions, that's maybe even more awesome.
Chapter \ref{prelim} is dedicated to bringing anyone up to speed so that they can understand and try to solve the problems that follow. Admittidely, however, that is a really tall order, and it might not even be practical,\footnote{To adequately fill in the necessary background for approaching these puzzles, you might just need to \dots write the lecture notes and textbooks for several undergraduate math courses. Also some of these problems are plenty crazy even for those who have studied undergraduate math (speaking as someone who has). Take for example August 2023's Single-Cross 2:
\begin{quote}
  Consider 3-space (i.e.\ $\mathbf{R}^3$) partitioned into a grid of unit cubes with faces defined by the planes of all points with at least one integer coordinate. For a fixed positive real number $D$, a random line segment of length $D$ (chosen uniformly in location and orientation) is placed in this cubic lattice.

  What length $D$ maximizes the probability that the endpoints of the segment lie in orthogonally adjacent unit cubes (that is, the segment crosses exactly one integer-coordinate plane), and what is this maximal probability? Give your answer as a \textbf{comma-separated pair of values to 10 significant places} (e.g.\ ``1.234567891,0.2468135792'').
\end{quote}} so it will almost certainly fall short. In fact, in it's current state, I don't know if Chapter \ref{prelim} is helpful at all. But with any luck that will change in the future!

Anyway, if you're not so hung up on learning background info and just want to jump into a puzzle, head on over to Chapter \ref{statement}---I'll see you there!
% END OF FILE 'preface.tex'
% START OF FILE 'preliminaries.tex'
\chapter{Preliminaries}\label{prelim}
Hey, a heads up: this section includes exercises, and it is expected that you do them. But don't worry;\footnote{Be happy.} the exercises are short and simple and their solutions are in Appendix \ref{answers}. Give each exercise an honest attempt before consulting its solution, though; you'll learn more this way, and you'll learn faster.

\section{A half-start}
\begin{quote}
\emph{Natural numbers were created by God, everything else is the work of men} — Kronecker (1823–1891)
\end{quote}
You might not think about it like this, but the very first thing you learned in math was how to count---as in ``there are three
apples over here and two over there, and if we put them together, there are five apples.'' The numbers involved in this basic form of counting are aptly called the \emph{natural numbers}. One is a natural number, and so is two and three and four and so on.\footnote{9001 is a natural number, too.}

But, as almost everyone already knows, these aren't the only numbers. Some numbers are \dots \emph{\ \!unnatural}, if history is any measure. Zero and negative numbers, for instance, were adopted well after the naturals. And it isn't too hard to imagine why: neither of them can be thought of in the same straight-forward, tangible way as a natural number (good luck pointing at zero or negative three apples). It is necessary to think about these numbers a little differently; we need to recognize that each is a unique leap in abstraction: zero represents the absence of a quantity rather than its multiplicity; and a negative number (in its most straight-forward light) represents a change (a \emph{reduction}) in number rather than a number outright.

The history of now-familiar numbers is fascinating but not the focus of what follows, the bare minimum ideas for pondering the puzzles. The hope of this mini historical detour was to shed light on how even the simplest ideas are worth revisiting, that in doing so we might see them in a new light and gain a new appreciation for them. What follows is a good deal of ``basic'' math which I hope the experienced and uninitiated alike will consider and respect rather than skim or skip. As G.\ P\'olya puts it, ``the advanced reader who skips parts that appear too elementary may miss more than the less advanced reader who skips parts that appear too complex.''

\section{Sets}
A set is a collection of distinct items or ``elements.'' Sets are written by listing their elements inside curly braces and can be represented by variables. Here are some example sets:
\begin{enumerate}[label = (\arabic*)]
\item $\set{terrier, poodle, Godzilla}$,
\item\label{set2} $A = \set{a, \dots, z}$,
\item\label{set3} $\N = \set{1, 2, 3, \dots}$.
\end{enumerate}
Ellipses `{$\dots$}' indicate implied elements that are tedious or impossible to write explicitly, as in items \ref{set2} and \ref{set3}, respectively. These examples also show that sets can be finite or infinite.\footnote{In fact, there are---in an abstract but rigorous sense---infinitely many and ever-growing kinds of infinities. But this is beyond our scope. For more on this, look up Cantor's Theorem.} The number of elements in a set $S$, known as its cardinality, is denoted $|S|$. For example, $|A| = 26$.

\begin{exercise}\label{distinct}
What is $|\{3, 3, 1, 2, 1, 3\}|$?
\end{exercise}

\subsection{The Cartesian product}
The Cartesian product of two sets $S$ and $T$ is the set of all ordered pairs with the first element from $S$ and the second from $T$, denoted $S \times T$.
For example, \[\set{1, 2} \times \set{a, b, c} = \set{(1, a), (1, b), (1, c), (2, a), (2, b), (2, c)}.\]
\begin{exercise}\label{cartesianques}
For finite $S$ and $T$, what is $|S \times T|$ in terms of $|S|$ and $|T|$?
\end{exercise}

\subsection{Bounds of sets}
A set may have lower (or upper) bounds when its elements can be ordered. For example, consider the set $H = \set{0, 1, 2, 3}$. The number $-100$ is a lower bound of $H$ because it is less than or equal to every element of $H$. However, there are greater lower bounds, such as $-99$, $-98$, and so on.
\begin{exercise}\label{lowerbound}
What is the greatest lower bound of $H$? Why?
\end{exercise}
We call the greatest lower bound of a set its \textit{infimum}. The minimum of a set is very similar: it is the least element \emph{contained} in a set.
\begin{exercise}\label{reallower}
Are the infimum and minimum of a set always the same? If they aren't, give an example.
\end{exercise}

\section{Binary Trees}
Figure \ref{tree}: the circles are called nodes, and the lines connecting them are called edges. The topmost node is called the root, and the bottom nodes are called leaves. The root has two descendants (hence the adjective ``binary''), each of which has two descendants of their own and so on. The depth of a tree is the number of edges between the root and any of its leaves; this tree's depth is four.

\begin{figure}[ht]
\centering
\begin{tikzpicture}
[
    scale = .65,
    level 1/.style={sibling distance=60mm, level distance = \levelheight},
    level 2/.style={sibling distance=30mm, level distance = \levelheight},
    level 3/.style={sibling distance=15mm, level distance = \levelheight},
    level 4/.style={sibling distance=7.5mm, level distance = \levelheight},
    state/.style={circle,minimum size =.35cm, draw}
]
	\node[state] {}
		child {node[state] {}
			child {node[state] {}
				child {node[state]{}
					child {node[state]{}}
					child {node[state]{}}
				}
				child {node[state]{}
					child{node[state]{}}
					child{node[state]{}}
				}
			}
			child {node[state] {}
				child {node[state]{}
					child {node[state]{}}
					child {node[state]{}}
				}
				child {node[state]{}
					child{node[state]{}}
					child{node[state]{}}
				}
			}
		}
		child {node[state] {}
		    	child {node[state] {}
				child {node[state]{}
					child {node[state]{}}
					child {node[state]{}}
				}
				child {node[state]{}
					child{node[state]{}}
					child{node[state]{}}
				}
			}
		    	child {node[state] {}
				child {node[state]{}
					child {node[state]{}}
					child {node[state]{}}
				}
				child {node[state]{}
					child{node[state]{}}
					child{node[state]{}}
				}
			}
		};
\end{tikzpicture}

\caption{An example binary tree.}
\label{tree}
\end{figure}
\begin{exercise}
How many leaves does a binary tree of depth $n$ have?
\end{exercise}

\section{Probability}
An event is \emph{random} when it cannot be predicted with complete certainty. The probability of an event $E$ is its likelihood of occurring, denoted $P(E)$. For example, the probability of rolling a 6 on a standard six-sided die (denoted ``d6'') is 1/6.

If an event is guaranteed to happen, its probability is 1. So for any event $E$, $0 \leq P(E) \leq 1$.
\begin{exercise}
Why? \emph{Hint:} This is not a trick question.
\end{exercise}
For convenience, we can denote the outcome of random events with variables. For example, let $D_n$ be the roll of a d$n$ (a standard $n$-sided die); then, again, $P(D_6 = 6) = 1/6$.

However, finding the probability of an event can be challenging when the context is unclear (e.g., what is the probability you think of the number $42$?). Therefore, it's helpful to keep in mind the set of all possible outcomes (relevant to an event), called the \emph{sample space} and denoted $\Omega$. For example, $\Omega = \set{1, 2, 3, 4, 5, 6}$ when rolling a d6: each number represents the outcome that the d6 rolls that value.

When $\Omega$ is finite and each event is equally likely, finding probabilities usually reduces to counting outcomes. Specifically:
\begin{equation}\label{countprob}
P(E) = \frac{\text{number of outcomes where $E$ occurs}}{\text{total number of outcomes}}.
\end{equation}
\begin{exercise}
What is $P(D_6 \text{ is prime})$?\footnote{A prime number is only divisible by 1 and itself. But the first prime number is 2, not 1, because excluding 1 simplifies many theorems about primes.}
\end{exercise}

If outcomes are not equally likely, we can modify the sample space by assigning each event a number of outcomes proportional to its likelihood. For instance, if we have a weighted d4 where each number $x$ comes up $(10\cdot x)\%$ of the time, this would be equivalent to a d10 with the number 1 on one face, 2 on two faces, and so on. 
\begin{exercise}\label{magiccoin}
  Imagine two coins, each with sides labelled 1 and 2. One coin, $F$, is fair,\footnote{(meaning each outcome is equally likely)} while the other one, $M$, is half-magic: if $F = 1$, then $M = 1$; otherwise, $M$ is fair. You flip $F$, then $M$. Describe a weighted die whose outcomes match the random variable $F + M$.

\noindent\emph{Hint}: apply equation \eqref{countprob} and the preceding paragraph.
\end{exercise}


\subsubsection*{Independence.}
Two random events are \emph{independent} if neither gives information about the likelihood of the other. For example, the first roll of a d6 gives no information about its second roll, so the two rolls are independent, unlike $M$ and $F$ in Exercise \ref{magiccoin}. Independent events are almost always easier to reason about than non-independent (i.e., dependent) events.\footnote{Compare the probability that the 100th roll of a d6 is 6 with the probability that the 100th roll of a \emph{special} d6 (identical to a standard one, except if you ever roll more than 20 of a number, it is replaced by the number 1) is 6.}

\subsubsection*{Multiple events.}
Events are usually composed of subevents. There are three fundamental operations that create new events from existing ones.
\begin{enumerate}[label = (\arabic*)]
\item Negation: The event where $E$ does not occur is written $\neg E$.
\item\label{intersection} Intersection:
The event where both $E$ and $F$ occur is written $\displaystyle E \cap F$.
\item Union:  The event where $E$ or $F$ (or both) occur is written $E \cup F$.
\end{enumerate}

\begin{exercise}\label{q:neg}
If $P(E) = p$, what is $P(\neg E)$?
\end{exercise}
\begin{exercise}\label{q:cap}
Suppose events $E$ and $F$ are independent. Can you guess what $\displaystyle P(E \cap F)$ is? \emph{Hint}: Recall the Cartesian product.
\end{exercise}
\begin{exercise}\label{q:cup}
Let $E$ and $F$ be any two events. What is $P(E \cup F)$ in terms of $P(E)$, $P(F)$, and $\displaystyle P(E \cap F)$? \emph{Hint}: Consider the event of rolling an even or prime number with a d6, and draw a Venn diagram.
\end{exercise}

\section{Polynomials}
\subsubsection*{In general.}
A polynomial is an expression made up of indeterminants where the only operations involved are addition, subtraction, and multiplication. For example $x^2 - x - 1$ is a polynomial made up of three terms and one indeterminant. This is a much clearer way of representing ``the product of a value with itself, subtracted by itself and 1,'' wouldn't you say?\footnote{We are lucky: mathematics was largely written in words before Ren\'e Descartes popularized symbolic notation in his ``La G\'eom\'etrie," published in 1637.}

The degree of a polynomial is the highest power of any indeterminant. For example, $x^3 - 7x + y$ has degree 3. A polynomial is called quadratic if its degree is 2.

\subsubsection*{Zeros and quadratic polynomials.}
The zeros of a polynomial are, unsurprisingly, the values of the variable(s) that make the polynomial equal to zero. For example, the zeros of $x^2 - 5x + 6 = (x-3)(x-2)$ are $x = 3$ and $x = 2$. Finding the zeros of a linear (degree 1) polynomial is trivial, but less so for a quadratic.

\begin{exercise}\label{completesquare}
Find a formula for the zeros of a quadratic $p(x) = ax^2 + bx + c$ in terms of its coefficients, $a$, $b$, and $c$.\footnote{The coefficients $a$, $b$, and $c$ are not to be confused with indeterminants: they represent only one possible value, unlike indeterminants that represent several possible values. In the earlier polynomial, $a = 1$, $b = -5$ and $c = 6$.} \emph{Hint}: Can you turn $p(x)$ into a form that looks like $(x + d)^2 = e$?\footnote{$d$ and $e$ are not indeterminants; they are particular values, known as \emph{constants}  (like $a$, $b$, and $c$).}
\end{exercise}
% END OF FILE 'preliminaries.tex'
% START OF FILE 'tree-edge-triage.tex'
\chapter{Tree-edge Triage}\label{statement}
Aaron and Beren are playing a game that takes place within an infinite binary tree. At the start of the game every edge in the tree is independently labeled $A$ with probability $p$  and $B$ otherwise.\footnote{On average, then, $p$ represents the proportion of edges in the tree that are labeled $A$.} Both players are able to inspect all (infinitely many) of these labels $\ldots$ somehow. Then, starting with Aaron at the root, the players alternate turns moving a shared token down the tree (each turn the active player selects from the two descendants of the current node and moves the token along the edge to that node). If the token ever traverses an edge labeled $B$, Beren wins the game. Otherwise, Aaron wins.\footnote{This is the August 2024 puzzle, BTW, which can be found at \url{https://www.janestreet.com/puzzles/tree-edge-triage-index/}}

\begin{figure}[ht]
\centering
\begin{tikzpicture}
[
    scale = 1.1,
    level 1/.style={sibling distance=42mm, level distance = 1.2cm},
    level 2/.style={sibling distance=22mm, level distance = 2.1cm},
    level 3/.style={sibling distance = 10mm, level distance = 1cm},
    level 4/.style={sibling distance = 5mm, level distance = .9cm},
    ellip/.style={text height = 4mm},
    game/.style={circle, draw, minimum size = .35cm, fill = none},
    whole/.style={draw=black, very thick, inner sep = 25pt, text height = 0mm},
    sub/.style={dashed, draw=black, very thick, inner sep=15pt}, % Style for grouping
    ssub/.style={draw = black, very thick, inner sep = 3pt},
    arrow/.style={draw, ->, color = RoyalBlue, line width = 1.7pt},
]

% Draw the binary tree with an edge label
\node[game] (root) {}
	child {node[game](11) {}
		child {node[game] {}
			child {node[game] {}
				child {node[ellip] {\bigdots}}
				child {node[ellip] {\bigdots}}
			edge from parent node [left] {$A$}
				}
			child {node[game] {}
				child {node[ellip] {\bigdots}}
				child {node[ellip] {\bigdots}}
			edge from parent node [right] {$B$}
				}
	edge from parent node [left] {$A$}
			}
		child {node[game](22) {}
			child {node[game] {}
				child {node[ellip] {\bigdots}}
				child {node[ellip] {\bigdots}}
			edge from parent node [left] {$B$}
				}
			child {node[game](34) {}
				child {node[ellip] {\bigdots}}
				child {node[ellip] {\bigdots}}
			edge from parent node [right] {$B$}
				}
		edge from parent node [right] {$A$}
			}
	edge from parent node [above] {$A$}
	}
	child {node[game] {}
		child {node[game] {}
			child {node[game] {}
				child {node[ellip] {\bigdots}}
				child {node[ellip] {\bigdots}}
			edge from parent node [left] {$B$}
				}
			child {node[game] {}
				child {node[ellip] {\bigdots}}
				child {node[ellip] {\bigdots}}
			edge from parent node [right] {$A$}
				}
		edge from parent node [left] {$B$}
			}
		child {node[game] {}
			child {node[game] {}
				child {node[ellip] {\bigdots}}
				child {node[ellip] {\bigdots}}
			edge from parent node [left] {$B$}
				}
			child {node[game](s) {}
				child {node[ellip] {\bigdots}}
				child {node[ellip] {\bigdots}}
			edge from parent node [right] {$A$}
				}
		edge from parent node [right] {$A$}
			}
	edge from parent node [above] {$B$}
	};
	\draw[arrow] (root) -- (11);
	\draw[arrow] (11) -- (22);
	\draw[arrow] (22) -- (34);
\end{tikzpicture}

\caption{An example game: Aaron chooses to go left to avoid immediate defeat. However, after Beren goes right, Aaron is forced to choose one of two $B$-edges, and Beren wins.}
\label{exGame}
\end{figure}

\begin{question}\label{infimum}
What is the infimum of the set of probabilities $p$ for which Aaron has a nonzero probability of winning the game?
\end{question}
This question is kind of dense for the sake of rigor that isn't that important right now. So here's another way to put it: what is the smallest average proportion of edges in the tree labelled $A$ such that Aaron still has a chance of winning?

\section{Solution}\label{solution}
Rather than tackling the entire infinite problem right away (which could be overwhelming), we first consider finite games restricted to finite binary trees. For these games we define the following events:
\begin{itemize}
\item Let $X_n$ denote Aaron wins a game that starts on \textbf{his} turn and lasts $n$ rounds.
\item Let $Y_n$ denote Aaron wins a game that starts on \textbf{Beren's} turn and lasts $n$ rounds.
\item Let $A_{\text{e}}$ denote edge $e$ is labelled $A$.
\end{itemize}
Now let's find some probabilities. The simplest one is $P(A_{e})$, which, by definition, is $p$ for any edge $e$. Next, it is clear that $X_1$ occurs when at least one of the two edges from the root, $e_1$ or $e_2$, is labeled $A$. Thus,
\begin{align*}
P(X_1) &= P(A_{e_1} \cup A_{e_2})\\
&= P(A_{e_1}) + P(A_{e_2})- P(A_{e_1} \cap A_{e_2})\\
&= 2p - p^2.
\end{align*}
Similarly, $Y_1$ occurs when both edges from the root are labelled $A$ (otherwise, Beren wins). Thus,
\[P(Y_1) = P(A_{e_1} \cap A_{e_2}) = p^2.\]
Why stop with $P(X_1)$ and $P(Y_1)$? Let's try to find $P(X_2)$. We could apply the same reasoning that got us $P(X_1)$, noting that $X_2$ occurs when either $e_1$ and its subsequent edges are labelled $A$ or $e_2$ and its subsequent edges are labelled A. But this approach becomes unwieldy as we consider more and more rounds. It would be nice if we could express $P(X_2)$ using what we already found---namely, $P(X_1)$ and $P(Y_1)$. And indeed, we can! The key is to realize that our events are composed of identical sub-events (see Figure \ref{subgames}), thanks to all the edges being labelled independently.
\begin{figure}[ht]
\centering
\begin{tikzpicture}
[
scale = 1,
level 1/.style={sibling distance=42mm, level distance = 1.2cm},
level 2/.style={sibling distance=18mm, level distance = 1.5cm},
level 3/.style={sibling distance = 8mm, level distance = 1cm},
ellip/.style={text height = 4mm},
game/.style={circle, draw, minimum size = .35cm, fill = none},
whole/.style={draw=black, thick, inner sep = 12pt, text height = 0mm},
sub/.style={dashed, draw=black, thick, inner sep=7pt}, % Style for grouping
ssub/.style={draw = black, thick, inner sep =2pt},
sssub/.style={dashed, draw = black, thick, inner sep = 3pt},
arrow/.style={draw, ->, color = RoyalBlue, line width = 1.7pt},
]

% Draw the binary tree with an edge label
\node[game] (root) {}
	child {node[game](11) {}
		child {node[game](21){}
			child {node[ellip](31) {\bigdots}
				}
			child {node[ellip](32) {\bigdots}
				}
			}
		child {node[game](22) {}
			child {node[ellip](33) {\bigdots}
				}
			child {node[ellip](34) {\bigdots}
				}
			}
	}
	child {node[game](12) {}
		child {node[game](23){}
			child {node[ellip](35) {\bigdots}
				}
			child {node[ellip] (36){\bigdots}
				}
			}
		child {node[game](24){}
			child {node[ellip](37) {\bigdots}
				}
			child {node[ellip](38) {\bigdots}
				}
			}
	};
\node[whole, label=above:{$X_n$}, fit = (root)(31)(38)] {};
\node[sub, label=above:{$Y_{n-1}$}, fit = (11)(21)(22)(31)(34)]{};
\node[sub, label=above:{$Y_{n-1}$}, fit = (12)(23)(24)(35)(38)]{};
\node[ssub, label = above:{$X_{n-2}$}, fit = (21)(31)(32)]{};
\node[ssub, label = above:{$X_{n-2}$}, fit = (22)(33)(34)]{};
\node[ssub, label = above:{$X_{n-2}$}, fit = (23)(35)(36)]{};
\node[ssub, label = above:{$X_{n-2}$}, fit = (24)(37)(38)]{};
\end{tikzpicture}

\caption{The sub-events of $X_n$}
\label{subgames}
\end{figure}

Thus, $X_2$ occurs if $e_1$ is labelled $A$ and Aaron \emph{wins the rest of the game after traversing $e_1$} ($Y_1$) or $e_2$ is labelled $A$ and Aaron \emph{wins the rest of the game after traversing $e_2$} ($Y_1$). That is,
\begin{align*}
P(X_2) &= P((A_{e_1} \cap Y_1) \cup (A_{e_2} \cap Y_1))\\
&= P(A_{e_1} \cap Y_1) + P(A_{e_2} \cap Y_1) - P((A_{e_1} \cap Y_1) \cap (A_{e_2} \cap Y_1))\\
&= 2pP(Y_1) - p^2P(Y_1)^2.
\end{align*}
Similarly,
\begin{align*}
P(Y_2) &= P((A_{e_1} \cap X_1) \cap (A_{e_2} \cap X_1))\\ 
P(Y_2) &= p^2P(X_{1})^2.
\end{align*}

But wait a second. Was there anything stopping us just now from writing $X_n$ instead of $X_2$ and $Y_{n-1}$ instead of $Y_1$ or $Y_n$ instead of $Y_2$ and $X_{n-1}$ instead of $X_1$? Absolutely not! This means that
\begin{gather}
P(X_n) = 2pP(Y_{n-1}) - p^2P(Y_{n-1})^2\text{ and}\label{xn}\\
P(Y_n) = p^2P(X_{n-1})^2.\label{yn}
\end{gather}

\begin{figure}[ht]
\centering
\begin{tikzpicture} \begin{axis}[
	colormap/cool,
	xlabel=$x$,
	ylabel=$p$,
	zlabel = $z$,
	zmin = -.15,
	zmax = 1,
    	title={$p^6x^4-2p^3x^2 + x$},
	view = {235}{30},
    	domain=0:1,
	y domain =0:1
]
\addplot3 [surf,thick,samples=25] {y^6*x^4 - 2*y^3*x^2 + x};
\end{axis}
\end{tikzpicture}
\caption{something}
\end{figure}

We now return to the infinite game. To answer Question \ref{infimum}, we need to determine for which values of $p$ $P(X_\infty) > 0$. First, let's find $P(X_\infty)$. If we take the limit as $n$ approaches infinity in \eqref{xn} then substitute in the value of $P(Y_n)$ from \eqref{yn}, we get
\begin{align*}
P(X_\infty) &= 2pP(Y_{\infty-1}) - p^2P(Y_{\infty-1})^2\\
&= 2p\Big(p^2P(X_{\infty-2})^2\Big) - p^2\Big(p^2P(X_{\infty-2})^2\Big)^2\\
&= 2p^3P(X_{\infty-2})^2 - p^6P(X_{\infty-2})^4.
\end{align*}
$P(X_\infty) = P(X_{\infty-2})$, and we still don't know what $P(X_\infty)$ is, so let it be $x$. Then we have $x = 2p^3x^2 - p^6x^4$ or, after rearranging terms,
\begin{equation*}
x(p^6x^3 - 2p^3x + 1) = 0.
\end{equation*}
We ignore the trivial solution $x = 0$ and set our sights on $p^6x^3 - 2p^3x + 1 = 0$. Let's remind ourselves what this equation represents. The possible values of $x$ that satisfy this equation are the possible values of $P(X_\infty)$. This is why we ignored $x=0$; we already know that Aaron's probability of winning can be $0$ (set $p = 0$, for example). 

The discriminant for a cubic polynomial of the form \[x^3 + ax^2 + bx + c\] is \[a^2b^2 + 18abc - 4b^3 - 4a^3c - 27c^2.\] Therefore
\begin{align*}
\Delta &= 0 + 0 -4b^3 - 0 - 27c^2\\
&=-4(-2p^{-3})^3 - 27(p^{-6})^2\\
&=2^5p^{-9} - 27p^{-12}.
\end{align*}

Setting $\Delta = 0$ and solving for $p$, we find that
\begin{align*}
2^5p&^{-9} = 27p^{-12}\\
2^5p&^3 = 3^3\\
p&^3 = 3^3\cdot 2^{-5}\\
p& = 3\cdot 2^{\frac{-5}{3}},
\end{align*}
which is our final answer.
% END OF FILE 'tree-edge-triage.tex'
% START OF FILE 'noodle.tex'
\chapter{Connecting Noodles}
There are $n$ noodles in a bowl. Two noodle ends are chosen uniformly at random (from all available ends)
and connected until none are left.

\begin{question*}
  How many loops form on average, and
  what is the probability that all the noodles connect to form one large loop? Can you find closed formulas for these expressions?
  \end{question*}

\section{Solution}
Each time we connect two noodle ends either a loop forms or it doesn't (and we make a longer noodle). In either case, two ends are removed,
so the effective number of noodles decreases by one. Therefore, with $n$ starting noodles, two ends will be connected exactly $n$ times, selecting
from the ends of $n$, $n-1$, $n-2$, etc. noodles each time.

We can find the probability a loop forms when there are $k$ noodles, $P(L_k)$, a few ways. The simplest is to notice that there are $2k-1$ available ends after the first is chosen, and that there's only one second end which will form a loop---the other end of
the first end's noodle---so $P(L_k)=\frac{1}{2k-1}$.

Alternatively, we can see that exactly $k$ out of the $\binom{2k}{2}$ possible noodle-end connections form a loop, so
\begin{equation}\label{plk}
P(L_k)=\frac{k}{\binom{2k}{2}} = \frac{k}{\frac{2k!}{2!(2k-2)!}} = \frac{k}{\frac{2k(2k-1)}{2}} = \frac{1}{2k-1}.
\end{equation}

Figure \ref{fig1} provides a visualization of noodle-end connections for $k\in\{2,4\}$.

\begin{figure}[h]
  \centering
\drawDiagram{1}
\drawDiagram{3}
\caption{Diagrams of all connections between two (left) and four (right) noodles. The gray doubled lines are the noodles, filled-in black circles
  are noodle ends, doubled lines are connections that form a loop, and dotted lines are connections that don't form loop. Knowing $\binom{8}{2} = 28$ and counting the doubled lines in the right diagram, we see that $P(L_4)=\sfrac{4}{28}=\sfrac{1}{7}$. We can also see more plainly from the left diagram that $P(L_2)=\sfrac{2}{6}=\sfrac{1}{3}$.
}
\label{fig1}
\end{figure}

\subsection*{Expected number of loops}
Let $X$ be the number of loops that form. Then $X = \sum_{k=1}^{n} L_k$, where
\begin{equation*}
L_k = \begin{cases}
  1 &\text{if a loop forms when there are $k$ noodles}\\
  0 &\text{otherwise}.
  \end{cases}
\end{equation*}
The average number of loops is of course $\expectation{X}$, and since $\expectation{X + Y} = \expectation{X}+\expectation{Y}$ (for all random variables) and
$\expectation{L_k} = P(L_k)$,
\[
\expectation{\sum_{k=1}^n L_k} = \sum_{k=1}^n \expectation{L_k} = \sum_{k=1}^n \frac{1}{2k-1} \sim \frac{1}{2}\ln n.
\]

\subsection*{Probability that one large loop forms}
The only way one large loop forms is if each time two
ends are connected\footnote{(Other than when there is only one noodle.)} a loop \textit{doesn't} form.

The probability a loop doesn't form when there are $k$ noodles is just
\[
1-P(L_k)=1-\frac{1}{2k-1}=\frac{2k-2}{2k-1},
\]
and since the selections
of noodle ends are independent, the probability that $L_n = 0$ and $L_{n-1} = 0$ and $L_{n-2} = 0$ and so on is just
\[
\prod_{k=2}^{n}\frac{2k-2}{2k-1} = \prod_{i=1}^{n-1}\frac{2i}{2i+1}.
\]

\subsection*{Finding a closed formula}
We've found the desired probability, but we'll go on to find a closed expression for it, starting with the numerator. By factoring out $2^{n-1}$ we get
\begin{equation}\label{numerator}
\prod_{i=1}^{n-1}2i = 2^{n-1}(n-1)!,
\end{equation}
which we can use to rewrite the denominator:
\begin{align*}
  \prod_{i=1}^{n-1}2i+1 
  &= 1\cdot3\cdot5\cdot\cdots\cdot(2n-1)
  = \frac{1\cdot2\cdot3\cdot\cdots\cdot(2n-2)}{2\cdot4\cdot6\cdot\cdots\cdot(2n-2)}\\
  &= \frac{(2n-2)!}{\prod_{i=1}^{n-1}2i}
  = \frac{(2n-2)!}{2^{n-1}(n-1)!} && \text{from \eqref{numerator}.}
\end{align*}

Substituting back in the rewritten numerator and denominator, we have the closed formula:
\[
\frac{(2^{n-1}(n-1)!)^2}{(2n-2)!}.
\]

\newpage
\subsection*{For fun}
Behold: Figure \ref{big}.
\begin{figure}[H]
  \centering
  \resizebox{!}{450pt}{\drawDiagram{15}}
  \caption{A diagram of all connections between $16$ noodles.}\label{big}
  \end{figure}
% END OF FILE 'noodle.tex'
% START OF FILE 'answers.tex'
\allowdisplaybreaks

\appendix

\chapter{Answers to the Exercises}\label{answers}

\renewcommand{\thechapter}{\arabic{chapter}}

\begin{answer}
If $P(E) = p$, then $P(\neg E) = 1 - p$.
\end{answer}

%% \begin{answer}%two
%%   Answer two.
%% \end{answer}

%% \begin{answer}%three
%%   Answer three.
%% \end{answer}

%% \begin{answer}%four
%%   Answer four.
%% \end{answer}

%% \begin{answer}%five
%%   Answer five.
%% \end{answer}

%% \begin{answer}%six
%%   Answer six.
%% \end{answer}

%% \begin{answer}%seven
%%   Answer seven.
%% \end{answer}

%% \begin{answer}%eight
%%   Answer eight.
%% \end{answer}

%% \begin{answer}%nine
%%   Answer nine.
%% \end{answer}

%% \begin{answer}%ten
%%   Answer ten.
%% \end{answer}

%% \begin{answer}%eleven
%%   Answer eleven.
%% \end{answer}

%% \begin{answer}%twelve
%%   Answer twelve.
%% \end{answer}


\begin{exercise*}[\ref{q:cap}]
When $E$ and $F$ are independent, finding $P(E \cap F)$ is straightforward because the sample space is simply the set of all pairs of possible outcomes from $E$ and $F$, i.e., $\Omega = \Omega_E \times \Omega_F$, where $\Omega_G$ is the sample space of an event $G$. As a result,
\begin{equation*}
P(E \cap F) = \frac{\#(E \cap F)}{|\Omega_E \times \Omega_F|}
= \frac{\#(E) \cdot \#(F)}{|\Omega_E| \cdot |\Omega_F|}
= P(E) \cdot P(F),
\end{equation*} 
where $\#(G)$ denotes the number of outcomes where event $G$ occurs (in its sample space). See Figure \ref{product} for an example.
\end{exercise*}

\begin{figure}[ht]
\centering
\begin{tikzpicture}
[
scale = 1.25,
v/.style = {Apricot, line width = 9mm},
h/.style = {Salmon, line width = 7mm},
rect/.style = {very thick}
]
\foreach \i in {1, 2, 3, 4, 5}{
	\foreach \j in {1, 2, 3, 4}{
		\coordinate (hs\j) at (.5, \j);
		\coordinate (he\j) at (5.35,\j);
		\coordinate (vs\i) at (\i, .75);
		\coordinate (ve\i) at (\i, 4.3);
	}
}
\begin{scope}[blend mode=screen]
\filldraw[v] (vs2) rectangle (ve2);
\filldraw[v] (vs3) rectangle (ve3);
\filldraw[v] (vs5) rectangle (ve5);
\filldraw[h] (hs2) rectangle (he2);
\filldraw[h] (hs4) rectangle (he4);
\end{scope}
\foreach \i in {1, 2, 3, 4, 5}{
	\foreach \j in {1, 2, 3, 4}{
		\coordinate (\i\j) at (\i, \j);
		\node (n\i\j) at (\i\j) {(\i, \j)};
		\coordinate (obl\i\j) at (\i -.375, \j -.275);
		\coordinate (otr\i\j) at (\i + .375, \j + .275);
	} 
}
\foreach \i in {1, 2, 3, 4, 5}{
\node at (\i, 4.65) {\i};
}
\foreach \i in {1, 2, 3, 4}{
\node at (.25, \i) {\i};
}
\draw[rect] (obl24) rectangle (otr24);
\draw[rect] (obl54) rectangle (otr54);
\draw[rect] (obl34) rectangle (otr34);
\draw[rect] (obl22) rectangle (otr22);
\draw[rect] (obl32) rectangle (otr32);
\draw[rect] (obl52) rectangle (otr52);
\end{tikzpicture}

\caption{The elements of $\Omega_{D_5} \times \Omega_{D_4}$. Let $X$ be the event where $D_5$ is prime, and let $Y$ be the event where $D_4$ is even. The outcomes where $X \cap Y$ occurs are boxed, and we can verify that $P(X \cap Y) = 6/20$ matches $P(X) \cdot P(Y) = \frac{3}{5} \cdot \frac{2}{4}$.}
\label{product}
\end{figure}


\begin{exercise*}[\ref{q:cup}]
Venn diagram properties shown in Figure \ref{vennd} reveal that for any two events $E$ and $F$, \[P(E \cup F) = P(E) + P(F) - P(E \cap F).\]
\end{exercise*}

\begin{figure}[ht]
\centering
\begin{tikzpicture}[scale = 1.15]
    % Define colors for the sets
    \begin{scope}[blend mode=overlay]
        % Draw the first set (A)
        \fill[Periwinkle] (-1,0) circle (1.5);
        % Draw the second set (B)
        \fill[Dandelion] (1,0) circle (1.5);
    \end{scope}
    
    \node at (-1.15, .7) (4){{\large 4}};
    \node at (-1.25, -.7) (6){{\large 6}};
    \node at (0, 0) (2) {{\large 2}};
    \node at (1.15, .7) (3){{\large 3}};
    \node at (1.25, -.7) (5){{\large 5}};
    \node at (-1,1.75) (E){{\large $E$}};
    \node at (2.5, 1.5) (1){{\large 1}};
    \node at (1,1.75) (F){{\large $F$}};
    \node at (0,-1.75) (EandF){{\large $E \cap F$}};
    \node at (0, -.7) (point){};
    \node at (-2.75, 0) (leftbox){};
    \node at (2.75, 0) (rightbox){};
    
    % Draw outlines for the sets
    \draw[thick] (-1,0) circle (1.5) node(Ev) {}; % Set A
    \draw[thick] (1,0) circle (1.5) node(Fv) {}; % Set B
    \draw[->, dotted, thick] (EandF) -- (point);
    \node[draw, inner sep = 3pt, very thick, fit = (Ev)(Fv)(EandF)(leftbox)(rightbox)(E)(F), label = above:{{\LARGE $\Omega$}}]{};
\end{tikzpicture}

\caption{A Venn diagram of two events in d6's sample space. $E$ is the event that the roll is even, and $F$ is the event that the roll is prime. So $P(E \cup F) = P(E) + P(F) - P(E \cap F) = 3/6 + 3/6 - 1/6 = 5/6$.}
\label{vennd}
\end{figure}

\begin{exercise*}[{\ref{completesquare}}] We first notice that
\[(x + h)^2 = x^2 + 2hx + h^2\] so \[\left(x + \frac{b}{2a}\right)^2 = x^2 + \frac{b}{a}x + \frac{b^2}{4a^2}.\] Then
\begin{align*}
ax^2 + bx + c &= 0\\
x^2 + \frac{b}{a}x + \frac{c}{a} &= 0\\
x^2 + \frac{b}{a}x + \frac{b^2}{4a^2} + \frac{c}{a} &= \frac{b^2}{4a^2}\\
\left(x + \frac{b}{2a}\right)^2 + \frac{c}{a} &= \frac{b^2}{4a^2}\\
\left(x + \frac{b}{2a}\right)^2 &= \frac{b^2-4ac}{4a^2}\\
x + \frac{b}{2a} &= \pm\sqrt{\frac{b^2}{4a^2}}\\
x &= \frac{-b \pm \sqrt{b^2 - 4ac}}{2a}.
\end{align*}
\end{exercise*}
% END OF FILE 'answers.tex'
\end{document}
