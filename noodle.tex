\documentclass{book}
\usepackage{amsthm}
\usepackage{amssymb}
\usepackage{appendix}
\usepackage{titlesec}
\usepackage{amsmath}
\usepackage{graphicx}
\usepackage[shortlabels]{enumitem}
\usepackage[dvipsnames]{xcolor}
\usepackage{tikz}
\usetikzlibrary{cd, fit, shapes.geometric, positioning}
\usepackage{pgfplots}
\pgfplotsset{compat=1.18}

\usepackage{hyperref}

\theoremstyle{definition}
\newtheorem{question}{Question}
\newtheorem*{definition*}{Definition}
\newtheorem{exercise}{Exercise}[chapter]
\newtheorem*{exercise*}{Answer}
\newtheorem{answer}{Answer}[chapter]

\newtheoremstyle{colonstylebf}{}{}{\normalfont}{}{\bfseries}{:}{.5em}{}
\theoremstyle{colonstylebf}
\newtheorem*{question*}{Q}

\newcommand{\set}[1]{\{#1\}}
\newcommand{\N}{\mathbb{N}}
\newcommand{\spvdots}{\hspace{5pt} \vdots \hspace{5pt} }
\newcommand*{\levelheight}{11mm}
\newcommand*{\bigdots}{{\Large $\vdots$}}

\titleformat{\subsubsection}[runin]
  {\normalfont\normalsize\bfseries} % Format of the title
  {\thesubsubsection} % Label
  {.5em} % Separation between label and title text
  {} % Before-code
  [] % After-code
  
  \titleformat{\title}[block]
  {\normalfont\Huge\scshape}
  {\thetitle}
  {1em}
  {}
  []

\begin{document}
\chapter{Connecting Noodles}
There are $n$ noodles in a bowl. Two noodle ends are chosen uniformly at random (from all available ends)
and connected until none are left.

\begin{question*}
  How many loops form on average, and
  what is the probability that all the noodles connect to form one large loop? Can you find closed formulas for these expressions?
  \end{question*}

\section{Solution}
Each time we connect two noodle ends either a loop forms or it doesn't (and we make a longer noodle). In either case, two ends are removed,
so the effective number of noodles decreases by one. Therefore, with $n$ starting noodles, two ends will be connected exactly $n$ times, selecting
from the ends of $n$, $n-1$, $n-2$, etc. noodles each time.

We can find the probability a loop forms when there are $k$ noodles, $P(L_k)$, a few ways. The simplest is to notice that there are $2k-1$ available ends after the first is chosen, and that there's only one second end which will form a loop---the other end of
the first end's noodle---so $P(L_k)=\frac{1}{2k-1}$.

Alternatively, we can see that exactly $k$ out of the $\binom{2k}{2}$ possible noodle-end connections form a loop, so
\begin{equation}\label{plk}
P(L_k)=\frac{k}{\binom{2k}{2}} = \frac{k}{\frac{2k!}{2!(2k-2)!}} = \frac{k}{\frac{2k(2k-1)}{2}} = \frac{1}{2k-1}.
\end{equation}

Figure \ref{fig1} provides a visualization of noodle-end connections for $k\in\{2,4\}$.

\begin{figure}[h]
  \centering
\drawDiagram{1}
\drawDiagram{3}
\caption{Diagrams of all connections between two (left) and four (right) noodles. The gray doubled lines are the noodles, filled-in black circles
  are noodle ends, doubled lines are connections that form a loop, and dotted lines are connections that don't form loop. Knowing $\binom{8}{2} = 28$ and counting the doubled lines in the right diagram, we see that $P(L_4)=\sfrac{4}{28}=\sfrac{1}{7}$. We can also see more plainly from the left diagram that $P(L_2)=\sfrac{2}{6}=\sfrac{1}{3}$.
}
\label{fig1}
\end{figure}

\subsection*{Expected number of loops}
Let $X$ be the number of loops that form. Then $X = \sum_{k=1}^{n} L_k$, where
\begin{equation*}
L_k = \begin{cases}
  1 &\text{if a loop forms when there are $k$ noodles}\\
  0 &\text{otherwise}.
  \end{cases}
\end{equation*}
The average number of loops is of course $\expectation{X}$, and since $\expectation{X + Y} = \expectation{X}+\expectation{Y}$ (for all random variables) and
$\expectation{L_k} = P(L_k)$,
\[
\expectation{\sum_{k=1}^n L_k} = \sum_{k=1}^n \expectation{L_k} = \sum_{k=1}^n \frac{1}{2k-1} \sim \frac{1}{2}\ln n.
\]

\subsection*{Probability that one large loop forms}
The only way one large loop forms is if each time two
ends are connected\footnote{(Other than when there is only one noodle.)} a loop \textit{doesn't} form.

The probability a loop doesn't form when there are $k$ noodles is just
\[
1-P(L_k)=1-\frac{1}{2k-1}=\frac{2k-2}{2k-1},
\]
and since the selections
of noodle ends are independent, the probability that $L_n = 0$ and $L_{n-1} = 0$ and $L_{n-2} = 0$ and so on is just
\[
\prod_{k=2}^{n}\frac{2k-2}{2k-1} = \prod_{i=1}^{n-1}\frac{2i}{2i+1}.
\]

\subsection*{Finding a closed formula}
We've found the desired probability, but we'll go on to find a closed expression for it, starting with the numerator. By factoring out $2^{n-1}$ we get
\begin{equation}\label{numerator}
\prod_{i=1}^{n-1}2i = 2^{n-1}(n-1)!,
\end{equation}
which we can use to rewrite the denominator:
\begin{align*}
  \prod_{i=1}^{n-1}2i+1 
  &= 1\cdot3\cdot5\cdot\cdots\cdot(2n-1)
  = \frac{1\cdot2\cdot3\cdot\cdots\cdot(2n-2)}{2\cdot4\cdot6\cdot\cdots\cdot(2n-2)}\\
  &= \frac{(2n-2)!}{\prod_{i=1}^{n-1}2i}
  = \frac{(2n-2)!}{2^{n-1}(n-1)!} && \text{from \eqref{numerator}.}
\end{align*}

Substituting back in the rewritten numerator and denominator, we have the closed formula:
\[
\frac{(2^{n-1}(n-1)!)^2}{(2n-2)!}.
\]

\newpage
\subsection*{For fun}
Behold: Figure \ref{big}.
\begin{figure}[H]
  \centering
  \resizebox{!}{450pt}{\drawDiagram{15}}
  \caption{A diagram of all connections between $16$ noodles.}\label{big}
  \end{figure}
\end{document}
