\documentclass{book}
\usepackage{amsthm}
\usepackage{amssymb}
\usepackage{appendix}
\usepackage{titlesec}
\usepackage{amsmath}
\usepackage{graphicx}
\usepackage[shortlabels]{enumitem}
\usepackage[dvipsnames]{xcolor}
\usepackage{tikz}
\usetikzlibrary{cd, fit, shapes.geometric, positioning}
\usepackage{pgfplots}
\pgfplotsset{compat=1.18}

\usepackage{hyperref}

\theoremstyle{definition}
\newtheorem{question}{Question}
\newtheorem*{definition*}{Definition}
\newtheorem{exercise}{Exercise}[chapter]
\newtheorem*{exercise*}{Answer}
\newtheorem{answer}{Answer}[chapter]

\newtheoremstyle{colonstylebf}{}{}{\normalfont}{}{\bfseries}{:}{.5em}{}
\theoremstyle{colonstylebf}
\newtheorem*{question*}{Q}

\newcommand{\set}[1]{\{#1\}}
\newcommand{\N}{\mathbb{N}}
\newcommand{\spvdots}{\hspace{5pt} \vdots \hspace{5pt} }
\newcommand*{\levelheight}{11mm}
\newcommand*{\bigdots}{{\Large $\vdots$}}

\titleformat{\subsubsection}[runin]
  {\normalfont\normalsize\bfseries} % Format of the title
  {\thesubsubsection} % Label
  {.5em} % Separation between label and title text
  {} % Before-code
  [] % After-code
  
  \titleformat{\title}[block]
  {\normalfont\Huge\scshape}
  {\thetitle}
  {1em}
  {}
  []

\begin{document}
\allowdisplaybreaks

\appendix

\chapter{Answers to the Exercises}\label{answers}

\renewcommand{\thechapter}{\arabic{chapter}}

\begin{answer}
If $P(E) = p$, then $P(\neg E) = 1 - p$.
\end{answer}

%% \begin{answer}%two
%%   Answer two.
%% \end{answer}

%% \begin{answer}%three
%%   Answer three.
%% \end{answer}

%% \begin{answer}%four
%%   Answer four.
%% \end{answer}

%% \begin{answer}%five
%%   Answer five.
%% \end{answer}

%% \begin{answer}%six
%%   Answer six.
%% \end{answer}

%% \begin{answer}%seven
%%   Answer seven.
%% \end{answer}

%% \begin{answer}%eight
%%   Answer eight.
%% \end{answer}

%% \begin{answer}%nine
%%   Answer nine.
%% \end{answer}

%% \begin{answer}%ten
%%   Answer ten.
%% \end{answer}

%% \begin{answer}%eleven
%%   Answer eleven.
%% \end{answer}

%% \begin{answer}%twelve
%%   Answer twelve.
%% \end{answer}


\begin{exercise*}[\ref{q:cap}]
When $E$ and $F$ are independent, finding $P(E \cap F)$ is straightforward because the sample space is simply the set of all pairs of possible outcomes from $E$ and $F$, i.e., $\Omega = \Omega_E \times \Omega_F$, where $\Omega_G$ is the sample space of an event $G$. As a result,
\begin{equation*}
P(E \cap F) = \frac{\#(E \cap F)}{|\Omega_E \times \Omega_F|}
= \frac{\#(E) \cdot \#(F)}{|\Omega_E| \cdot |\Omega_F|}
= P(E) \cdot P(F),
\end{equation*} 
where $\#(G)$ denotes the number of outcomes where event $G$ occurs (in its sample space). See Figure \ref{product} for an example.
\end{exercise*}

\begin{figure}[ht]
\centering
\begin{tikzpicture}
[
scale = 1.25,
v/.style = {Apricot, line width = 9mm},
h/.style = {Salmon, line width = 7mm},
rect/.style = {very thick}
]
\foreach \i in {1, 2, 3, 4, 5}{
	\foreach \j in {1, 2, 3, 4}{
		\coordinate (hs\j) at (.5, \j);
		\coordinate (he\j) at (5.35,\j);
		\coordinate (vs\i) at (\i, .75);
		\coordinate (ve\i) at (\i, 4.3);
	}
}
\begin{scope}[blend mode=screen]
\filldraw[v] (vs2) rectangle (ve2);
\filldraw[v] (vs3) rectangle (ve3);
\filldraw[v] (vs5) rectangle (ve5);
\filldraw[h] (hs2) rectangle (he2);
\filldraw[h] (hs4) rectangle (he4);
\end{scope}
\foreach \i in {1, 2, 3, 4, 5}{
	\foreach \j in {1, 2, 3, 4}{
		\coordinate (\i\j) at (\i, \j);
		\node (n\i\j) at (\i\j) {(\i, \j)};
		\coordinate (obl\i\j) at (\i -.375, \j -.275);
		\coordinate (otr\i\j) at (\i + .375, \j + .275);
	} 
}
\foreach \i in {1, 2, 3, 4, 5}{
\node at (\i, 4.65) {\i};
}
\foreach \i in {1, 2, 3, 4}{
\node at (.25, \i) {\i};
}
\draw[rect] (obl24) rectangle (otr24);
\draw[rect] (obl54) rectangle (otr54);
\draw[rect] (obl34) rectangle (otr34);
\draw[rect] (obl22) rectangle (otr22);
\draw[rect] (obl32) rectangle (otr32);
\draw[rect] (obl52) rectangle (otr52);
\end{tikzpicture}

\caption{The elements of $\Omega_{D_5} \times \Omega_{D_4}$. Let $X$ be the event where $D_5$ is prime, and let $Y$ be the event where $D_4$ is even. The outcomes where $X \cap Y$ occurs are boxed, and we can verify that $P(X \cap Y) = 6/20$ matches $P(X) \cdot P(Y) = \frac{3}{5} \cdot \frac{2}{4}$.}
\label{product}
\end{figure}


\begin{exercise*}[\ref{q:cup}]
Venn diagram properties shown in Figure \ref{vennd} reveal that for any two events $E$ and $F$, \[P(E \cup F) = P(E) + P(F) - P(E \cap F).\]
\end{exercise*}

\begin{figure}[ht]
\centering
\begin{tikzpicture}[scale = 1.15]
    % Define colors for the sets
    \begin{scope}[blend mode=overlay]
        % Draw the first set (A)
        \fill[Periwinkle] (-1,0) circle (1.5);
        % Draw the second set (B)
        \fill[Dandelion] (1,0) circle (1.5);
    \end{scope}
    
    \node at (-1.15, .7) (4){{\large 4}};
    \node at (-1.25, -.7) (6){{\large 6}};
    \node at (0, 0) (2) {{\large 2}};
    \node at (1.15, .7) (3){{\large 3}};
    \node at (1.25, -.7) (5){{\large 5}};
    \node at (-1,1.75) (E){{\large $E$}};
    \node at (2.5, 1.5) (1){{\large 1}};
    \node at (1,1.75) (F){{\large $F$}};
    \node at (0,-1.75) (EandF){{\large $E \cap F$}};
    \node at (0, -.7) (point){};
    \node at (-2.75, 0) (leftbox){};
    \node at (2.75, 0) (rightbox){};
    
    % Draw outlines for the sets
    \draw[thick] (-1,0) circle (1.5) node(Ev) {}; % Set A
    \draw[thick] (1,0) circle (1.5) node(Fv) {}; % Set B
    \draw[->, dotted, thick] (EandF) -- (point);
    \node[draw, inner sep = 3pt, very thick, fit = (Ev)(Fv)(EandF)(leftbox)(rightbox)(E)(F), label = above:{{\LARGE $\Omega$}}]{};
\end{tikzpicture}

\caption{A Venn diagram of two events in d6's sample space. $E$ is the event that the roll is even, and $F$ is the event that the roll is prime. So $P(E \cup F) = P(E) + P(F) - P(E \cap F) = 3/6 + 3/6 - 1/6 = 5/6$.}
\label{vennd}
\end{figure}

\begin{exercise*}[{\ref{completesquare}}] We first notice that
\[(x + h)^2 = x^2 + 2hx + h^2\] so \[\left(x + \frac{b}{2a}\right)^2 = x^2 + \frac{b}{a}x + \frac{b^2}{4a^2}.\] Then
\begin{align*}
ax^2 + bx + c &= 0\\
x^2 + \frac{b}{a}x + \frac{c}{a} &= 0\\
x^2 + \frac{b}{a}x + \frac{b^2}{4a^2} + \frac{c}{a} &= \frac{b^2}{4a^2}\\
\left(x + \frac{b}{2a}\right)^2 + \frac{c}{a} &= \frac{b^2}{4a^2}\\
\left(x + \frac{b}{2a}\right)^2 &= \frac{b^2-4ac}{4a^2}\\
x + \frac{b}{2a} &= \pm\sqrt{\frac{b^2}{4a^2}}\\
x &= \frac{-b \pm \sqrt{b^2 - 4ac}}{2a}.
\end{align*}
\end{exercise*}
\end{document}
