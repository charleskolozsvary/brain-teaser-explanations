\usepackage{graphicx}
\usepackage{amsthm}
\usepackage{amssymb}
\usepackage{appendix}
\usepackage{titlesec}
\usepackage{amsmath}
\usepackage{graphicx}
\usepackage[shortlabels]{enumitem}
\usepackage[dvipsnames]{xcolor}
\usepackage{tikz}
\usetikzlibrary{cd, fit, shapes.geometric, positioning}
\usepackage{pgfplots}
\pgfplotsset{compat=1.18}

\usepackage{xfrac}
\usepackage{float}
\usepackage{hyperref}

\theoremstyle{definition}
\newtheorem{question}{Question}
\newtheorem*{definition*}{Definition}
\newtheorem{exercise}{Exercise}[chapter]
\newtheorem*{exercise*}{Answer}
\newtheorem{answer}{Answer}[chapter]

\newtheoremstyle{colonstylebf}{}{}{\normalfont}{}{\bfseries}{:}{.5em}{}
\theoremstyle{colonstylebf}
\newtheorem*{question*}{Q}

\newcommand*{\set}[1]{\{#1\}}
\newcommand*{\N}{\mathbb{N}}
\newcommand{\spvdots}{\hspace{5pt} \vdots \hspace{5pt} }
\newcommand*{\levelheight}{11mm}
\newcommand*{\bigdots}{{\Large $\vdots$}}

\titleformat{\subsubsection}[runin]
  {\normalfont\normalsize\bfseries} % Format of the title
  {\thesubsubsection} % Label
  {.5em} % Separation between label and title text
  {} % Before-code
  [] % After-code
  
  \titleformat{\title}[block]
  {\normalfont\Huge\scshape}
  {\thetitle}
  {1em}
  {}
  []


%% for noodle.tex (pretty gross, but...)
\gdef\true{1}
\gdef\false{0}

\gdef\drawPair#1#2#3#4#5{% 
\expandafter\ifx\csname#1#2#3#4\endcsname\true\else%
\expandafter\ifx\csname#3#4#1#2\endcsname\true\else%
\edef\first{#1#2}%
\edef\second{#3#4}%
\ifx\first\second\else
\draw[#5] (#1,#2) to (#3,#4);% 
\global\expandafter\let\csname#1#2#3#4\endcsname\true%
\global\expandafter\let\csname#3#4#1#2\endcsname\true%
\fi\fi\fi}

\gdef\drawDiagram#1{\begin{tikzpicture}
  \def\n{#1}
  \pgfmathparse{\n/2}
  \xdef\midpoint{\pgfmathresult}
  \foreach \y in {0,...,\n}{
    \foreach \x in {-1,1}{
      \foreach \j in {0,...,\n}{
        \foreach \i in {-1,1}{
          \expandafter\gdef\csname\x\y\i\j\endcsname{0}
          \expandafter\gdef\csname\i\j\x\y\endcsname{0}          
        }
      }
    }
  }
  \foreach \y in {0,...,\n}{
    \draw[double,gray!50] (-1,\y) -- (1,\y);
  }

  \foreach \y in {0,...,\n}{
    \foreach \x in {-1,1}{
      \foreach \j in {0,...,\n}{
        \foreach \i in {-1,1}{
          \pgfmathparse{\x == -1 && \i == -1}
          \edef\onLeft{\pgfmathresult}
          \pgfmathparse{((\y > \midpoint) && (\j > \midpoint)) && (\y == \j)}
          \edef\aboveAndAcross{\pgfmathresult}
          \pgfmathparse{\onLeft || \aboveAndAcross}
          \edef\bendLeft{\pgfmathresult}%
          %
          \pgfmathparse{\y==\j}
          \edef\across{\pgfmathresult}
          %
          \pgfmathparse{\y!=\j && \x == -\i}
          \edef\straight{\pgfmathresult}%
          %
          \ifx\straight\true%
          \drawPair\x\y\i\j{densely dotted}
          \else%
          \ifx\bendLeft\true%
              \ifx\across\true
                  \drawPair\x\y\i\j{bend left, double}%
              \else%
                  \drawPair\x\y\i\j{bend left,densely dotted}%
              \fi
          \else%
              \ifx\across\true
                  \drawPair\x\y\i\j{bend right, double}
              \else
                  \drawPair\x\y\i\j{bend right,densely dotted}
              \fi
          \fi\fi
        }
      }
    }
  }
  \foreach \y in {0,...,\n}{
    \foreach \x in {-1,1}{
      \filldraw (\x,\y) circle (.05);
      %% \draw[double] (\x,\y) circle (.04);      
    }
  }
\end{tikzpicture}}

\newcommand{\expectation}[1]{\mathbb{E}\left[#1\right]}
\let\ocdot\cdot
\newcommand{\newcdot}{\mathop{\ocdot}}
\let\cdot\newcdot
